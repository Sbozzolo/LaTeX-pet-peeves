\documentclass[]{article}
\usepackage{amsmath}
\usepackage{amssymb}
\usepackage{siunitx}
\renewcommand{\d}[1]{\ensuremath{\operatorname{d}\!{#1}}}

\DeclareSIUnit[]\sunmass{\text{\ensuremath{M_{\odot}}}}

\author{Gabriele Bozzola}
\title{What (not) to do in \LaTeX}
\date{}

\begin{document}
\maketitle
\tableofcontents

\section{Spaces}%
\label{sec:spaces}

TL;DR\@: add ``\textbackslash'' after periods in abbreviations. Add ``@'' before
periods in capitalized acronyms (e.g., TL;DR).

In \LaTeX, spaces have different lengths. For example, the space between words
is smaller than the space between sentences. This simple trick makes the
document more readable. However, how does the rendering engine understand what
kind of space to use? For that, a series of heuristics are used. For instance, a
period (aka, a full stop) is typically used to indicate the end of a sentence,
so it makes sense to put more space after this punctuation mark. However, this
is not infallible. Consider abbreviations like ``e.g.'', ``i.e.''. Here we have
periods that do no not indicate the end of a sentence, but it is impossible for
\LaTeX~to know that, so, too much space will be used. Consider the difference
between the following two sentences:

\noindent
There are many foods I like to eat in the summer. I like fruit, e.g. apples, bananas.\\
There are many foods I like to eat in the summer. I like fruit, e.g.\ apples, bananas.\\

There's a clear difference between the two sentences. The first one is incorrect.
The second one was obtained adding a backslash after the period:
\begin{verbatim}
I like fruit, e.g.\ apples, bananas.
\end{verbatim}

\LaTeX~behaves in the opposite way with capitalized letters. It is assumed that
periods indicate an acronym, e.g.~N.A.S.A., so \LaTeX~will assume that the
sentence is not completed when it ends with a capitalized letter and a period.
To inform the rendering engine that this is wrong, ``\textbackslash @'' has to
be used \emph{before} the period.

\begin{verbatim}
a very powerful GPU\@.
\end{verbatim}

\noindent
My computer does not have a very powerful GPU. This is not a big problem.\\
My computer does not have a very powerful GPU\@. This is not a big problem.\\

Typically, this does not result in very noticeable errors, but it is always
best to inform \LaTeX~of your intents to avoid rendering problems.

\end{document}
